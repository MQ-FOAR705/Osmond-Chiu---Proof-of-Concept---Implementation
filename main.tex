\documentclass[10pt,a4paper]{article}
\usepackage[utf8]{inputenc}

\usepackage[landscape,margin=1cm]{geometry}
\usepackage[english]{babel}


% colour themes to come. KnitR?

%-------------------------

\title{Instructions for Technology Deployment}
\author{Osmond Chiu}
\date{2019}
\input{cheatsheet-template.tex}



%--------------------------------------------------------------------------------
\begin{document}
\small

\maketitle
\thispagestyle{empty}
\scriptsize
\tableofcontents





%--------------------------------------------------------------
\section{Tools}


\begin{textbox}{Zotero}
 \href{https://zotero.org}{Zotero} \sep \href{https://zotero.org}{bibliographic software}

\includegraphics[width=\textwidth]{zotero.png}
\end{textbox}

\begin{textbox}{Hypothes.is}
 \href{https://hypothes.is/}{Hypothes.is} \sep \href{https://hypothes.is/}{Web annotation}

\includegraphics[width=\textwidth]{Hypothesis.png}

\end{textbox}

\begin{textbox}{OpenSemantic Search}
 \href{https://www.opensemanticsearch.org}{OpenSemantic Search} \sep \href{https://www.opensemanticsearch.org}{Search Engine and Open Source Text Mining & Text Analytics}

\includegraphics[width=\textwidth]{search.png}

\end{textbox}


%--------------------------------------------------------------

\section{Getting Started}



\begin{textbox}{Install Package}
test  \sep test \sep test \sep test

\bigskip

\green{Instructions}
\red{Package}
\begin{enumerate}
\item \textbf{Download} from the repository at \href{https://github.com/MQ-FOAR705/Osmond-Chiu---Proof-of-Concept---Implementation.git}{https://github.com/MQ-FOAR705/Osmond-Chiu---Proof-of-Concept---Implementation.git}
\item \textbf{Run} the downloaded package to begin installation
\item \textbf{Read} these instructions
\item \textbf{Tick} off completed tasks from checklist
\end{enumerate}

\end{textbox}

%--------------------------------------------------------------


\section{Zotero}
\subsection{Setup}
\begin{textbox}{Complete Installation}
test  \sep test \sep test \sep test

\bigskip

\green{Instructions}
\red{Zotero}
\begin{enumerate}
\item \textbf{Check} that Zotero installation file has been mounted if installation does not automatically pop-up
\item \textbf{Drag} the Zotero icon to 'Drag here to install'
\end{enumerate}

\end{textbox}

\begin{textbox}{Create account}
test  \sep test \sep test \sep test

\bigskip

\green{Instructions}
\red{Zotero}
\begin{enumerate}
\item \textbf{Open} browser
\item \textbf{Go to}  \href{https://www.zotero.org/user/login}{https://www.zotero.org/user/login}. 
\item \textbf{Click} Register for a Free Account.
\item \textbf{Choose} the username and email address you want to use for the account. 
\item \textbf{Click} Register and you’ll receive a confirmation email shortly. 
\item \textbf{Click} on the link in the confirmation email you receive to confirm your account set up.
\end{enumerate}

\end{textbox}


\begin{textbox}{Sync account}
test  \sep test \sep test \sep test

\bigskip

\green{Instructions}
\red{Zotero}
\begin{enumerate}
\item \textbf{Open} Zotero.
\item \textbf{Open} Preferences (via the Zotero menu) and select the Sync tab.
\item \textbf {Enter} your Zotero user name and password.
\item \textbf {Click} set up syncing.
\item \textbf {Check} both boxes under File Syncing and choose Zotero storage for My Library.
\item \textbf {Close} Preferences
\item \textbf {Click} the green circular arrow button at the top right corner of the Zotero program window to sync account with software.
\end{enumerate}

\end{textbox}

\subsection{Using Zotero With Other Programs}

\begin{textbox}{Link Zotero to LaTex}
test  \sep test \sep test \sep test

\bigskip

\green{Instructions}
\red{Zotero}
\begin{enumerate}
\item \textbf{Log} into Overleaf.
\item \textbf{Open} Preferences (via the Zotero menu) and select the Sync tab.
\item \textbf {Enter} your Zotero user name and password.
\item \textbf {Go} to Account Settings.
\item \textbf {Scroll} down the page to Zotero Integration.
\item \textbf {Click} on 'Link to Zotero' and you will be prompted to log into Zotero.org
\item \textbf {Enter} Zotero username and password.
\item \textbf {Create} a New Private Key.
\item \textbf {Click} on Accept Defaults and you will be automatically returned to your Overleaf account settings page.
\item \textbf {Scroll} down to the bottom of the page to check Zotero Integration.
\end{enumerate}

\end{textbox}

 
 \begin{textbox}{Import references}
test  \sep test \sep test \sep test

\bigskip

\green{Instructions}
\red{Zotero}
\begin{enumerate}
\item \textbf {Go} to Overleaf 
\item \textbf {Select} the project that you want to link to Zotero
\item \textbf {Select} New File from the top menu.
\item \textbf {Select} 'From Zotero', name the file for the project and choose a format (BibTeX or BibLaTeX).
\item \textbf {Go} to main.tex file and add your Zotero file (as a bib resource) to header with the code \begin{verbatim} \bibliographystyle{style}  \end{verbatim}.
\end{enumerate}

\end{textbox}

  \begin{textbox}{Insert references}
test  \sep test \sep test \sep test

\bigskip

\green{Instructions}
\red{Zotero}
\begin{enumerate}
\item \textbf{Insert} and select a Zotero citation by typing into the source code \begin{verbatim}\cite\end{verbatim}
\end{enumerate}

\end{textbox}

 \begin{textbox}{Generate bibliography}
test  \sep test \sep test \sep test

\bigskip

\green{Instructions}
\red{Zotero}
\begin{enumerate}
\item \textbf{Insert} your bibliography by typing into the source code \begin{verbatim}\bibliography{filename} \end{verbatim}.
\end{enumerate}

\end{textbox}


%--------------------------------------------------------------


\section{Hypothes.is}
\subsection{Setup}

\begin{textbox}{Create account}
test  \sep test \sep test \sep test

\bigskip

\green{Instructions}
\red{Hypothes.is}
\begin{enumerate}
\item \textbf{Open} browser
\item \textbf{Go to} \href{hypothes.is/signup}{hypothes.is/signup}. 
\item \textbf{Choose} a username, enter your email address, and create a password.
\item \textbf{Click} Sign Up and you will receive a confirmation email to active your account.
\item \textbf{Click} the link in the email received to validate your email and activate your Hypothes.is account
\end{enumerate}

\end{textbox}

\begin{textbox}{Add Chrome Extension}
test  \sep test \sep test \sep test

\bigskip

\green{Instructions}
\red{Hypothes.is}
\begin{enumerate}
\item \textbf{Go to} web.hypothes.is/start. 
\item \textbf{Click} click add-on for Chrome.
\item \textbf{Accept} the prompt in the pop up window to install the extension. 
\item \textbf{Consider} adding a bookmarklet if you do not have/use Chrome.
\end{enumerate}

\end{textbox}

\begin{textbox}{Add Bookmarklet}
test  \sep test \sep test \sep test

\bigskip

\green{Instructions}
\red{Hypothes.is}
\begin{enumerate}
\item \textbf{Go to} web.hypothes.is/start. 
\item \textbf{Drag} 'Hypothesis Bookmarklet' to the bookmarks bar, or right-click/control-click to bookmark the link.
\end{enumerate}

\end{textbox}

\subsection{Using Hypothes.is}

\begin{textbox}{Highlight}
test  \sep test \sep test \sep test

\bigskip

\green{Instructions}
\red{Hypothes.is}
\begin{enumerate}
\item \textbf{Go to} webpage or open a local file in a browser. 
\item \textbf{Click} on Hypothes.is icon in browswer toolbar or bookmarklet.
\item \textbf{Click} toggle or resize sidebar.
\item \textbf{Log} into your Hypothes.is account.
\item \textbf{Choose} whether you want the higlight to be public or in a private group .
\item \textbf{Select} text that you wish to highlight.
\item \textbf{Click} on the Highlight icon that appeals.

\end{enumerate}

\end{textbox}

\begin{textbox}{Annotate}
test  \sep test \sep test \sep test

\bigskip

\green{Instructions}
\red{Hypothes.is}
\begin{enumerate}
\item \textbf{Go to} webpage. 
\item \textbf{Click} on Hypothes'is icon in browswer toolbar or bookmarklet.
\item \textbf{Click} toggle or resize sidebar.
\item \textbf{Log} into your Hypothes.is account.
\item \textbf{Choose} whether you want the higlight to be public or in a private group .
\item \textbf{Select} text that you wish to annotate.
\item \textbf{Click} on the Annotate icon that appeals.
\item \textbf{Type} annotations in text box
\item \textbf{Type} tags for the annotation
\item \textbf{Choose} where to post the annotations (public or private group) and click 

\end{enumerate}

\end{textbox}


\begin{textbox}{Generate API token}
test  \sep test \sep test \sep test

\bigskip

\green{Instructions}
\red{Hypothes.is}
\begin{enumerate}
\item \textbf{Go to} Hypothes.is website. 
\item \textbf{Log} into your Hypothes.is account.
\item \textbf{Click} on gear icon.
\item \textbf{Click} on 'Developer.
\item \textbf{Click} Generate Your API token.
\item \textbf{Copy} token in text box for use.

\end{enumerate}

\end{textbox}


\section{OpenSemantic Search}
\subsection{Setup}


\begin{textbox}{Install Virtual Box}
test  \sep test \sep test \sep test

\bigskip

\green{Instructions}\red{OpenSemantic Search}

\begin{enumerate}
\item \textbf {Double-click} on the VirtualBox.pkg installer file displayed in that window.
\item \textbf {Select} where to install Oracle VM VirtualBox.

\end{enumerate}


\end{textbox}

\begin{textbox}{Install OpenSemantic Search}
test  \sep test \sep test \sep test

\bigskip

\green{Instructions}\red{OpenSemantic Search}
\begin{enumerate}
\item \textbf{Download} the virtual machine image Open Semantic Desktop Search. 
\item \textbf {Find} Oracle VM VirtualBox icon in the "Applications" folder in the Finder.
\item \textbf{Start} Virtual Box.
\item \textbf{Go} into the menu "File" start the option "Import Appliance".
\item \textbf{Choose} the downloaded appliance file and start the import Configuration of document folders
\item \textbf{Edit} the settings of the new virtual machine (choose the virtual machine in the left sidebar and click the Settings button in the top bar).
\item \textbf{Add} shared folders with documents.
\item \textbf{Active} the option Auto-mount.
\end{enumerate}

\end{textbox}

\begin{textbox}{Load OpenSemantic Search}
test  \sep test \sep test \sep test

\bigskip

\green{Instructions}\red{OpenSemantic Search}
\begin{enumerate}
\item \textbf{Download} 
\end{enumerate}

\end{textbox}

\begin{textbox}{Text search}
test  \sep test \sep test \sep test

\bigskip

\green{Instructions}\red{OpenSemantic Search}
\begin{enumerate}
\item \textbf{Download} 
\end{enumerate}

\end{textbox}


\begin{textbox}{Create tags}
test  \sep test \sep test \sep test

\bigskip

\green{Instructions}\red{OpenSemantic Search}
\begin{enumerate}
\item \textbf{Click} on Thesaurus in the topbar of the OpenSemantic Search interface.
\item \textbf{Select} Tag as Facet 
\item \textbf{Type} name of Tag 
\item \textbf{Save} tag 
 
\end{enumerate}

\end{textbox}

\begin{textbox}{Tag documents}
test  \sep test \sep test \sep test

\bigskip

\green{Instructions}\red{OpenSemantic Search}
\begin{enumerate}
\item \textbf{Search} for document you wish to tag
\item \textbf{Click} Tagging and annotation for this document in the search results
\item \textbf{Add} the pre-existing tags to the document 
\end{enumerate}

\end{textbox}


\begin{textbox}{Import from Hypothes.is}
test  \sep test \sep test \sep test


\bigskip

\green{Instructions}\red{OpenSemantic Search}\red{Hypothes.is}
\begin{enumerate}
\item \textbf{Repeat} previous instructions to find API token in Hypothes.is account 
\item \textbf{Don't} regenerate a new token
\item \textbf{Copy} API Token
\item \textbf{Open} loaded OpenSemantic Search
\item \textbf{Click} on Datasources in the topbar of the OpenSemantic Search interface
\item \textbf{Click} on Annotations (Hypothesis)
\item \textbf{Enter} copied API Token for Authorization
\end{enumerate}


\end{textbox}

\end{document}
















