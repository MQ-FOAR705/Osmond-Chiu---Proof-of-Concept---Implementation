\documentclass[10pt,a4paper]{article}
\usepackage[utf8]{inputenc}

\usepackage[landscape,margin=1cm]{geometry}
\usepackage[english]{babel}


% colour themes to come. KnitR?

%-------------------------

\title{Instructions for Technology Deployment}
\author{Osmond Chiu}
\date{2019}
\input{cheatsheet-template.tex}



%--------------------------------------------------------------------------------
\begin{document}
\small

\maketitle
\thispagestyle{empty}
\scriptsize
\tableofcontents





%--------------------------------------------------------------
\section{Tools}


\begin{textbox}{Zotero}
 \href{https://zotero.org}{Zotero} \sep \href{https://zotero.org}{bibliographic software}

\includegraphics[width=\textwidth]{zotero.png}
\end{textbox}

\begin{textbox}{Hypothes.is}
 \href{https://hypothes.is/}{Hypothes.is} \sep \href{https://hypothes.is/}{Web annotation}

\includegraphics[width=\textwidth]{Hypothesis.png}

\end{textbox}

\begin{textbox}{OpenSemantic Search}
 \href{https://www.opensemanticsearch.org}{OpenSemantic Search} \sep \href{https://www.opensemanticsearch.org}{Search Engine and Open Source Text Mining & Text Analytics}

\includegraphics[width=\textwidth]{search.png}

\end{textbox}


%--------------------------------------------------------------

\section{Getting Started}



\begin{textbox}{Install Package}
test  \sep test \sep test \sep test

\bigskip

\green{Instructions}
\red{Package}
\begin{enumerate}
\item \textbf{Download} the package
\item \textbf{Run} the package
\item \textbf{Install} the software
\item \textbf{Read} these instructions
\end{enumerate}

\end{textbox}

%--------------------------------------------------------------


\section{Zotero}
\subsection{Setup}
\begin{textbox}{Complete Installation}
test  \sep test \sep test \sep test

\bigskip

\green{Instructions}
\red{Zotero}
\begin{enumerate}
\item \textbf{Check} that Zotero installation file has been mounted if installation does not automatically pop-up
\item \textbf{Drag} the Zotero icon to 'Drag here to install'
\end{enumerate}

\end{textbox}

\begin{textbox}{Create account}
test  \sep test \sep test \sep test

\bigskip

\green{Instructions}
\red{Zotero}
\begin{enumerate}
\item \textbf{Go to}  \href{https://www.zotero.org/user/login}{https://www.zotero.org/user/login}. 
\item \textbf{Click} Register for a Free Account.
\item \textbf{Choose} the username and email address you want to use for the account. 
\item \textbf{Click} Register and you’ll receive a confirmation email shortly. 
\item \textbf{Click} on the link in the confirmation email you receive to confirm your account set up.
\end{enumerate}

\end{textbox}


\begin{textbox}{Sync account}
test  \sep test \sep test \sep test

\bigskip

\green{Instructions}
\red{Zotero}
\begin{enumerate}
\item \textbf{Open} Zotero.
\item \textbf{Open} Preferences (via the Zotero menu) and select the Sync tab.
\item \textbf {Enter} your Zotero user name and password.
\item \textbf {Click} set up syncing.
\item \textbf {Check} both boxes under File Syncing and choose Zotero storage for My Library.
\item \textbf {Close} Preferences
\item \textbf {Click} the green circular arrow button at the top right corner of the Zotero program window to sync account with software.
\end{enumerate}

\end{textbox}

\subsection{Using Zotero With Other Programs}

\begin{textbox}{Link Zotero to LaTex}
test  \sep test \sep test \sep test

\bigskip

\green{Instructions}
\red{Zotero}
\begin{enumerate}
\item \textbf{Log} into Overleaf.
\item \textbf{Open} Preferences (via the Zotero menu) and select the Sync tab.
\item \textbf {Enter} your Zotero user name and password.
\item \textbf {Go} to Account Settings.
\item \textbf {Scroll} down the page to Zotero Integration.
\item \textbf {Click} on 'Link to Zotero' and you will be prompted to log into Zotero.org
\item \textbf {Enter} Zotero username and password.
\item \textbf {Create} a New Private Key.
\item \textbf {Click} on Accept Defaults and you will be automatically returned to your Overleaf account settings page.
\item \textbf {Scroll} down to the bottom of the page to check Zotero Integration.
\end{enumerate}

\end{textbox}

 
Scroll back up to the top of the account settings page and click on 'Projects.' Then, from your 'Projects' page, select the project you want to connect to Zotero.
 
Once you are viewing a specific project, select 'New File' from the top right menu.


 
You will be prompted with a number of choices. Select 'From Zotero' and you will be prompted to name your file and to choose a format (BibTeX or BibLaTeX). Once you have named your file and picked a format, click on 'Create.' PLEASE NOTE: Your filename must end with .bib for it to work properly.


 
Add your Zotero file (as a bib resource) to header of your main.tex file.


 
Insert a Zotero citation as you would any other citation from a .bib file!



%--------------------------------------------------------------


\section{Hypothes.is}
\subsection{Setup}

\begin{textbox}{Create account}
test  \sep test \sep test \sep test

\bigskip

\green{Instructions}
\red{Hypothes.is}
\begin{enumerate}
\item \textbf{Go to} \href{hypothes.is/signup}{hypothes.is/signup}. 
\item \textbf{Choose} a username, enter your email address, and create a password.
\item \textbf{Click} Sign Up and you’ll receive a confirmation email shortly. 
\end{enumerate}

\end{textbox}

\begin{textbox}{Add Chrome Extension}
test  \sep test \sep test \sep test

\bigskip

\green{Instructions}
\red{Hypothes.is}
\begin{enumerate}
\item \textbf{Go to} web.hypothes.is/start. 
\item \textbf{Click} click add-on for Chrome.
\item \textbf{Accept} the prompt in the pop up window to install the extension. 
\item \textbf{Consider} adding a bookmarklet if you do not have/use Chrome.
\end{enumerate}

\end{textbox}

\begin{textbox}{Add Bookmarklet}
test  \sep test \sep test \sep test

\bigskip

\green{Instructions}
\red{Hypothes.is}
\begin{enumerate}
\item \textbf{Go to} web.hypothes.is/start. 
\item \textbf{Drag} 'Hypothesis Bookmarklet' to the bookmarks bar, or right-click/control-click to bookmark the link.
\end{enumerate}


\end{textbox}

\section{OpenSemantic Search}
\subsection{Setup}


\begin{textbox}{Install Virtual Box}
test  \sep test \sep test \sep test

\bigskip

\green{Instructions}\red{OpenSemantic Search}

\begin{enumerate}
\item \textbf {Double-click} on the VirtualBox.pkg installer file displayed in that window.
\item \textbf {Select} where to install Oracle VM VirtualBox.

\end{enumerate}


\end{textbox}

\begin{textbox}{Install OpenSemantic Search}
test  \sep test \sep test \sep test

\bigskip

\green{Instructions}\red{OpenSemantic Search}
\begin{enumerate}
\item \textbf{Download} the virtual machine image Open Semantic Desktop Search. 
\item \textbf {Find} Oracle VM VirtualBox icon in the "Applications" folder in the Finder.
\item \textbf{Start} Virtual Box.
\item \textbf{Go} into the menu "File" start the option "Import Appliance".
\item \textbf{Choose} the downloaded appliance file and start the import Configuration of document folders
\item \textbf{Edit} the settings of the new virtual machine (choose the virtual machine in the left sidebar and click the "settings" button in the top bar).
\item \textbf{Add} shared folders with documents.
\item \textbf{Active} the option Auto-mount.
\end{enumerate}

\end{textbox}

\end{document}
















